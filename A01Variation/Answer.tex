%! TeX encoding = UTF-8
%! TeX program = LuaLaTeX

\documentclass[english, nochinese]{pnote}
\usepackage[paper]{pdef}

\title{Answers to Assignments 1}
\author{Zhihan Li, 1600010653}
\date{October 22, 2018}

\begin{document}

\maketitle

\textbf{Poblem 1.} \textit{Answer.} Denote the center as $ \rbr{ i, j } $. For the straight line function $u$ (depicted in the left), norm of discretized gradient is
\begin{equation}
\norm{ \nabla u }_{ i, j } = \sqrt{ \rbr{u_x}_{ i, j }^2 + \rbr{u_y}_{ i, j } ^2 } = \sqrt{ \rbr{\frac{1}{ 2 h }}^2 + 0^2 } = \frac{1}{ 2 h }.
\end{equation}
For the inclined line $v$ (depicted in the right), the norm is
\begin{equation}
\norm{ \nabla v }_{ i, j } = \sqrt{ \rbr{v_x}_{ i, j }^2 + \rbr{v_y}_{ i, j } ^2 } = \sqrt{ \rbr{\frac{ 1 + \lambda }{ 4 h }}^2 + \rbr{\frac{ 1 + \lambda }{ 4 h }}^2 } = \frac{ \sqrt{2} \rbr{ 1 + \lambda } }{ 4 h }.
\end{equation}
As a result, $ \norm{ \nabla u }_{ i, j } = \norm{ \nabla v }_{ i, j } $ yields $ \lambda = \sqrt{2} - 1 $.

Similarly, the descritized Laplacian operator for $u$ and $v$ are
\begin{gather}
\rbr{ \Delta u }_{ i, j } = \frac{-1}{h^2}, \\
\rbr{ \Delta v }_{ i, j } = \frac{ -1 - 3 \lambda }{ 2 h^2 }
\end{gather}
and the condition for $ \rbr{ \Delta u }_{ i, j } = \rbr{ \Delta v }_{ i, j } $ is $ \lambda = 1 / 3 $.

\end{document}
